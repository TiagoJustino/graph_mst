% LaTeX document template.
% Author: Rudy Matela
% e-mail: rudy.matela@gmail.com
% date: 2008-11-08 23:20


\documentclass[12pt,a4paper]{article}
\usepackage[utf8]{inputenc}
\usepackage[brazil]{babel}
\usepackage{graphicx}
\usepackage{indentfirst}
\usepackage{verbatim}
\usepackage{a4wide}
\usepackage{sty/macros}

\title{Algoritmos para Árvore Geradora Mínima}
\author{ Renato Aguiar \and Tiago Justino }
\date{
    Prof. Marcos Negreiros \\
	Universidade Estadual do Ceará \\
	$\cdots$ \\
	\today
}


\begin{document}

\maketitle

\tableofcontents
\pagebreak

\section{Objetivos}

\begin{enumerate}
\item Mostre, implemente e avalie os algoritmos PRIM, KRUSKAL e BORUVKA da árvore
geradora mínima, usando estruturas de prioridade Fibonacci e Relações de
Equivalência. Mostre as diferenças entre os algoritmos avaliando o seu
comportamento conforme tipos específicos de grafos indicados pelo professor.
Estudando também a corretude dos métodos.

\item Desenvolva adaptações nos algoritmos da questão anterior, e agora avalie o
seu comportamento para as seguintes situações:

\begin{enumerate}
\item mostrar as k árvores garadoras mínimas (é possível que todas as k sejam
iguais em custo);

\item mostrar a menor árvore de custo positivo, e grau de vértices nunca maior
que $\gamma$ dado.

\item mostrar as menores subárvores com custo positivo, admitindo que nenhuma
subárvore excede um tamanho especificado dmax.

\item compor os itens b e c em uma única visualização (ativando as duas
restrições simultaneamente).
\end{enumerate}

Para todos os casos anteriores considere:


\item Projete, avalie a corretude, implemente e teste os seguintes algoritmos
para o problema da AGM com restrição de grau:

\begin{enumerate}
\item Um algoritmo de Programação Dinâmica;
\item Um algoritmo Exaustivo puro;
\item Um algoritmo Las Vegas
\end{enumerate}

\end{enumerate}

Apresente um relatório que mostra claramente o que foi feito em cada questão
acima e os algoritmos utilizados. Os algoritmos acima têm que ser
determinísticos e podem compor de forma hibrida tipos de diversas naturezas.

Para as questões acima, implementar algoritmos usando uma interface visual que
permita ver o a instância de teste e o resultado visual, custo da solução final
obtida, assim como o tempo gasto do processo.

Os grafos devem ser lidos do seguinte modo: GRAFO.TXT // Arquivo Texto

NV, TG, NE //Numero de Vertices, Tipo do Grafo (Euclidiano ou Simétrico), Numero
de elos se simétrico.

COX COY // Indice do Vertice e Coordenadas

$\cdots$

COX COY // Ultimo Vertice - NV

I J W(I,J) // I < J, Elo do grafo e Custo ** Se não tiver elos, esta parte
estará vazia

$\cdots$

K,L W(K,L) // Ultimo elo

PS.: 10 instâncias de teste são fornecidas pelo professor. As instâncias
euclidianas não possuem ligações. Caso alguma isntância não esteja no formato
padrão, ajuste-as para que estejam dentro do formato.

\section{Fundamentação Teórica}
%\subsection{Grafo} %548
%\subsection{Árvore}
\subsection{Minimum Spanning Tree} %584
Dado um grafo simétrico conexo $G(V,E)$ onde $V$ são os vértices e $E$ é um
conjunto de elos possíveis entre pares de vértices e para cada elo $(u, v) \in
E$, temos um peso $w(u, v)$ especificando o custo (ou a distância) para conectar
$u$ e $v$. Desejamos então encontrar um subconjunto aciclico $T \subseteq E$ que
conecte todos os vértices e cujo peso total $w(T) = \sum_{(u, v) \in T} w(u, v)$
seja minimizado. Uma vez que $T$ é aciclico e conecta todos os vertices, ele
deve formar uma árvore que chamaremos de árvore geradora (spanning tree), já que
cobre todos os vértices do grafo $G$\cite[p. 584]{cormen2001introduction}.

%QUADRADO2009_ini.dot kruskal_01_quadrado2009_mst.dot

A figura \ref{fig:kruskal_01_quadrado2009_mst} mostra um exemplo de uma MST para
o grafo \ref{fig:QUADRADO2009_ini}.

Neste trabalho iremos explorar os algoritmos de Prim, Kruskal e Boruvka que
podem ser implementados com complexidade $O(E lg V)$ utilizando heaps binarios.
Com a utilização de heaps de Fibonacci o algoritmo de Prim possui complexidade
$O(E + V lg V)$, o que pode apresentar um ganho de performance mas somente
quando $|E|$ for muito maior que $|V|$.

\subsection{Euclidian Minimum Spanning Tree}

%A Euclidean graph is a graph in which the vertices represent points in the
%plane, and the edges are assigned lengths equal to the Euclidean distance
%between those points. The Euclidean minimum spanning tree is the minimum
%spanning tree of a Euclidean complete graph. It is also possible to define
%graphs by conditions on the distances; in particular, a unit distance graph is
%formed by connecting pairs of points that are a unit distance apart in the
%plane. The Hadwiger–Nelson problem concerns the chromatic number of these
%graphs.

Um Grafo Euclidiano é um grafo cujos vértices representam pontos em um plano e
cujas arestas possuem peso igual a distância euclidiana entre esses pontos. A
EMST (Euclidian Minimum Spanning Tree - Árvore Geradora Mínima Euclidiana) é uma
MST para um grafo Euclidiano. 

%The simplest algorithm to find an EMST, given n points, is to actually construct
%the complete graph on n vertices, which has n(n-1)/2 edges, compute each edge
%weight by finding the distance between each pair of points, and then run a
%standard minimum spanning tree algorithm (such as the version of Prim's
%algorithm or Kruskal's algorithm) on it. Since this graph has Θ(n2) edges for n
%distinct points, constructing it already requires Ω(n2) time. This solution also
%requires Ω(n2) space to store all the edges.

%A better approach to finding the EMST in a plane is to note that it is a
%subgraph of every Delaunay triangulation of the n points, a much-reduced set of
%edges:

%1. Compute the Delaunay triangulation in O(n log n) time and O(n) space.
%Because the Delaunay triangulation is a planar graph, and there are no more than
%three times as many edges as vertices in any planar graph, this generates only
%O(n) edges.  2. Label each edge with its length.  3. Run a graph minimum
%spanning tree algorithm on this graph to find a minimum spanning tree. Since
%there are O(n) edges, this requires O(n log n) time using any of the standard
%minimum spanning tree algorithms such as Borůvka's algorithm, Prim's algorithm,
%or Kruskal's algorithm.

%The final result is an algorithm taking O(n log n) time and O(n) space.

%If the input coordinates are integers and can be used as array indices, faster
%algorithms are possible: the Delaunay triangulation can be constructed by a
%randomized algorithm in O(n log log n) expected time.[1] Additionally, since the
%Delaunay triangulation is a planar graph, its minimum spanning tree can be found
%in linear time by a variant of Borůvka's algorithm that removes all but the
%cheapest edge between each pair of components after each stage of the
%algorithm.[3] Therefore, the total expected time for this algorithm is O(n log
%log n).[1]

O algoritmo mais simples para encontrar uma EMST é: dados $n$ pontos, construir
o grafo completo nos $n$ vértices, que tem $n(n-1)/2$ arestas, computar o peso
de cada aresta encontrando a distância entre cada par de pontos e entâo executar
algum dos algoritmos de MST (Prim, Kruskal ou Boruvka). Como o grafo tem
$O(n^2)$ arestas para $n$ pontos distintos, a construção do grafo já requer
$O(n^2)$ de tempo e espaço.

%02_DONI22009_delaunay.dot 02_DONI22009_euclidean.dot

Uma abordagem mais eficiente é construir um grafo pela triangulação de Delaunay
sobre os $n$ pontos e então executar algum algoritmo de MST no grafo gerado uma
vez que uma EMST em um plano é um subgrafo da triangulação de Delaunay nos $n$
pontos. Uma prova disso pode ser encontrada em \cite{subramani-computational}.

A triangulação de Delaunay requer $O(n log n)$ de tempo e $O(n)$ de espaço e
gera um número de arestas da ordem de $n$. Como o número de arestas é da ordem
de $n$ a execução de um dos três algoritmos de MST citados levará tempo $O(n log
n)$

Em nosso trabalho de implementação utilizamos a biblioteca Triangle que pode ser
baixada no site http://www.cs.cmu.edu/~quake/triangle.html e utilizada
livremente para fins não comerciais.  A partir da entrada DONI22009 (apresentada
logo abaixo) foi gerado o grafo enclidiano mostrado em
\ref{fig:02_DONI22009_euclidean}. Em seguida a triangulação de Delaunay mostrada
em \ref{fig:02_DONI22009_delaunay}. Por último foi gerada a EMST
\ref{fig:kruskal_02_DONI22009_mst}.

\verbatiminput{DONI22009.txt}

\subsection{Degree Constrained Minimum Spanning Tree}

%One of the main goals in the design process is to reach total connectivity at
%minimum cost. Furthermore, additional constraints must be met, such as a
%restricted number of connections to a physical unit. Similar problems arise in
%the planning of road maps, where intersections can only be established among a
%small number of roads. In the field of integrated circuit design we are faced
%with the constraint that the number of wired connections to an electronic
%component is limited.  The production of the circuit board shall be at minimum
%cost.  Problems of this kind can be modelled as the degree-constrained minimum
%span- ning tree problem (or DCMST problem for short).

Um dos principais objetivos no processo de desenvolvimento de redes é atingir
uma conectividade completa à um custo mínimo. Constantemente, algumas outras
restrições devem ser satisfeitas, como um número restrito de conecções a uma
unidade física. Problemas similares surgem no desenvolvimento de circuitos
integrados, onde nos deparamos com uma restrição em relação ao número de fios
conectados a um componente eletrônico. Problemas dessa natureza podem ser
modelados como uma DCMST (Degree Constrained Minimum Spanning Tree - Árvore
Geradora Mínima com Restrição de Grau)\cite{behle2007primal}.

Em 1979, Garey \& Johnson provaram em \cite{garey1979computers} que o problema
da DCMST é NP-Completo. Isso pode ser mostrado com uma redução a partir do
problema do caminho Hamiltoniano. Devido a sua complexidade existem uma série de
proposta de algoritmos aproximativos construtivos e metaheurísticas para a
solução deste problema.

Em nosso trabalho implementamos dois algoritmos aproximativos constutivos para
esse problema. O primeiro deles é uma adaptação do prim, também conhecido como
d-prim que impede a entrada de uma aresta caso viole a restrição de
grau\cite{narula1980degree}. O segundo é uma adaptação do algoritmo de kruskal,
que, de forma semelhante ao primeiro, impede a entrada de uma aresta que viole a
restrição de grau.

\section{Aplicativo}
O aplicativo possui interface web. A tela inicial do aplicativo é bem simples:
você insere o arquivo e clica em calcular. Em seguida o aplicativo apresenta um
relatório mostrando o tempo gasto em cada algoritmo e os resultados obtidos.
Para cada algoritmo o link para uma imagem é disponibilizado. A imagem apresenta
o grafo obtido pelo algoritmo. Screenshoots do aplicativo são apresentados nas
figuras de \ref{fig:main_window} a \ref{fig:image}.

\section{Resultados}

Em nossa implementação, executando em um processador AMD Athlon(tm) 64 Processor
3000+ de 1000 MHz no sistema operacional GNU/Linux, distribuição Ubuntu x86\_64
versão lucid e kernel 2.6.32-24 conseguimos obter os seguintes resultados:

\begin{itemize}
\item Kruskal
\begin{itemize}
\item 01\_quadrado2009.txt
\begin{itemize}
\item Tempo gasto = 0.000000 
\item Peso total = 32
\end{itemize}
\item 02\_DONI22009.txt
\begin{itemize}
\item Tempo gasto = 0.010000 
\item Peso total = 106.40793336244902906 
\end{itemize}
\item 03\_serrinha2009.txt
\begin{itemize}
\item Tempo gasto = 0.020000 
\item Peso total = 15739.05 
\end{itemize}
\item 04\_palmeiras2008.txt
\begin{itemize}
\item Tempo gasto = 0.010000 
\item Peso total = 555.149999999999999995 
\end{itemize}
\item 05\_DONI12009.txt
\begin{itemize}
\item Tempo gasto = 0.060000 
\item Peso total = 255.900099222521051567 
\end{itemize}
\item 06\_DONI12009.txt
\begin{itemize}
\item Tempo gasto = 0.100000 
\item Peso total = 345.318666013446358693 
\end{itemize}
\end{itemize}
\item Prim 
\begin{itemize}
\item 01\_quadrado2009.txt
\begin{itemize}
\item Tempo gasto = 0.000000 
\item Peso total = 32
\end{itemize}
\item 02\_DONI22009.txt
\begin{itemize}
\item Tempo gasto = 0.010000 
\item Peso total = 106.40793336244902906 
\end{itemize}
\item 03\_serrinha2009.txt
\begin{itemize}
\item Tempo gasto = 0.000000 
\item Peso total = 15739.05 
\end{itemize}
\item 04\_palmeiras2008.txt
\begin{itemize}
\item Tempo gasto = 0.010000 
\item Peso total = 555.149999999999999995 
\end{itemize}
\item 05\_DONI12009.txt
\begin{itemize}
\item Tempo gasto = 0.020000 
\item Peso total = 255.900099222521051567 
\end{itemize}
\item 06\_DONI12009.txt
\begin{itemize}
\item Tempo gasto = 0.040000 
\item Peso total = 345.318666013446358692 
\end{itemize}
\end{itemize}
\item Boruvka
\begin{itemize}
\item 01\_quadrado2009.txt
\begin{itemize}
\item Tempo gasto = 0.000000
\item Peso total = 32
\end{itemize}
\item 02\_DONI22009.txt
\begin{itemize}
\item Tempo gasto = 0.000000
\item Peso total = 106.40793336244902906
\end{itemize}
\item 03\_serrinha2009.txt
\begin{itemize}
\item Tempo gasto = 0.020000
\item Peso total = 15739.05
\end{itemize}
\item 04\_palmeiras2008.txt
\begin{itemize}
\item Tempo gasto = 0.020000
\item Peso total = 555.149999999999999995
\end{itemize}
\item 05\_DONI12009.txt
\begin{itemize}
\item Tempo gasto = 0.060000
\item Peso total = 255.900099222521051567
\end{itemize}
\item 06\_DONI12009.txt
\begin{itemize}
\item Tempo gasto = 0.270000
\item Peso total = 345.318666013446358692
\end{itemize}
\end{itemize}
\end{itemize}

%\bibliographystyle{plain}
%\bibliographystyle{cell}
\bibliographystyle{alpha}
\bibliography{bib}

\section{Anexo: Lista de figuras}

\fig[.8]{main_window}{Tela Principal}
\fig[.8]{report}{Relatório}
\fig[.8]{image}{Imagem do grafo}
\fig[.6]{QUADRADO2009_ini}{Um grafo direcional}
\fig[.6]{kruskal_01_quadrado2009_mst}{Uma MST para \ref{fig:QUADRADO2009_ini}}
\fig[.6]{02_DONI22009_euclidean}{Um grafo Euclidiano}
\fig[.6]{02_DONI22009_delaunay}{Uma triangulação de Delaunay para o grafo
\ref{fig:02_DONI22009_euclidean}}
\fig[.6]{kruskal_02_DONI22009_mst}{Uma EMST paga o grafo Euclidiano
\ref{fig:02_DONI22009_euclidean}}

\end{document}


\documentclass{beamer}

%\usetheme{default}
%\usetheme{Montpellier}
\usetheme{split}

\usecolortheme{default}
%\usecolortheme{seagull}

\usepackage[utf8]{inputenc}
\usepackage[brazil]{babel}
\usepackage{graphicx}
\usepackage{verbatim}
\usepackage{moreverb}
\usepackage{multicol}
\usepackage{sty/macros}


\title[Template]{
	Template de apresentação em \LaTeX usando o pacote Beamer
}
\author[Fulano]{Fulano de Tal}
\institute[CCC - UUUU]{
	Prof. Dr. Cicrano das Tantas\\
	Disciplina de Apresentações Apresentativas\\
	Curso de Ciências Científicas\\
	Universidade Universitária dos Universitários Unidos\\
}
\date{\today}




\AtBeginSection[] {
	\framet[Tópicos]{
		\tableofcontents[currentsection,hideothersubsections]
	}
}

\begin{document}

\frame{\titlepage}

\setcounter{tocdepth}{1}
\framet[Tópicos]{ \tableofcontents }
\setcounter{tocdepth}{2}


\sframei{Slide-seção com tópicos}{
	\item Um tópico
	\item Outro tópico
	\item Mais outro
	\item Agora um tópico com sub-tópicos \ize{
		\item sub-tópico
		\item outro
	}
	\item Mais um tópico pra finalizar
}


\framei[Slide com tópicos e figura]{
	\item Uma figura vai a seguir

	\begin{center}
	\includegraphics[width=0.5\textwidth]{fig/euler}
	\end{center}

	\item É uma questão de gosto usar ou não o ambiente figure em apresentações
	\item O exemplo acima não usa
	\item Este slide não cria seção nem subseção
}


\ssframei{Slide sub-seção com um de cada vez}{
	\item<1-> Isso aparece primeiro
	\item<2-> Mais um slide e aparece isso
	\item<3-> Agora isso
}


\ssframet{Subseção com apenas texto}{
	Isto é um slide com apenas texto.  Lorem ipsum dolor sit amet, consectetur
	adipisicing elit, sed do eiusmod tempor incididunt ut labore et dolore
	magna aliqua. Ut enim ad minim veniam, quis nostrud exercitation ullamco
	laboris nisi ut aliquip ex ea commodo consequat. Duis aute irure dolor in
	reprehenderit in voluptate velit esse cillum dolore eu fugiat nulla
	pariatur. Excepteur sint occaecat cupidatat non proident, sunt in culpa qui
	officia deserunt mollit anim id est laborum.  Quidquid latine dictum sit
	altum viditur.
}


\sframed{Outro slide seção}{
	\item[isto] é um slide
	\item[que] demonstra
	\item[o uso] de descrições de itens
}


\framet[Slide com apenas texto]{
	Isto é um slide com apenas texto.  Lorem ipsum dolor sit amet, consectetur
	adipisicing elit, sed do eiusmod tempor incididunt ut labore et dolore
	magna aliqua. Ut enim ad minim veniam, quis nostrud exercitation ullamco
	laboris nisi ut aliquip ex ea commodo consequat. Duis aute irure dolor in
	reprehenderit in voluptate velit esse cillum dolore eu fugiat nulla
	pariatur. Excepteur sint occaecat cupidatat non proident, sunt in culpa qui
	officia deserunt mollit anim id est laborum.  Quidquid latine dictum sit
	altum viditur.
}


\end{document}




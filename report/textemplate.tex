% LaTeX document template.
% Author: Rudy Matela
% e-mail: rudy.matela@gmail.com
% date: 2008-11-08 23:20


\documentclass[12pt,a4paper]{article}
\usepackage[utf8]{inputenc}
\usepackage[brazil]{babel}
\usepackage{graphicx}
\usepackage{indentfirst}
\usepackage{verbatim}
\usepackage{a4wide}
\usepackage{sty/macros}

\title{Template de documento em \LaTeX\ Versão 2.3}
\author{ J. Cicrano \and C. Beltrano \and Fulano de Tal }
\date{
	Universidade Fdsa do Asdf\\
	Laboratório Foo Goo Bar\\
	$\cdots$ \\
	\today
}


\begin{document}


\maketitle

\begin{abstract}
Este documento é um template de documento em \LaTeX. The quick brown fox jumps over the lazy dog.
Lorem ipsum dolor sit amet, consectetur adipisicing elit, sed do eiusmod tempor
incididunt ut labore et dolore magna aliqua. Ut enim ad minim veniam, quis
nostrud exercitation ullamco laboris nisi ut aliquip ex ea commodo consequat.
Duis aute irure dolor in reprehenderit in voluptate velit esse cillum dolore eu
fugiat nulla pariatur. Excepteur sint occaecat cupidatat non proident, sunt in
culpa qui officia deserunt mollit anim id est laborum.
\end{abstract}

\tableofcontents
\pagebreak


\section{Introdução}

Introdução do template. The quick brown fox jumps over the lazy dog. Lorem
ipsum dolor sit amet, consectetur adipisicing elit, sed do eiusmod tempor
incididunt ut labore et dolore magna aliqua. Ut enim ad minim veniam, quis
nostrud exercitation ullamco laboris nisi ut aliquip ex ea commodo consequat.
Duis aute irure dolor in reprehenderit in voluptate velit esse cillum dolore eu
fugiat nulla pariatur. Excepteur sint occaecat cupidatat non proident, sunt in
culpa qui officia deserunt mollit anim id est laborum.


\section{Exemplos}


\subsection{Descrição}

A descrição é descrita abaixo:

\begin{description}
\item[Aspecto A:] Aspecto A.
\item[Aspecto B:] Aspecto B.
\item[Aspecto C:] Aspecto C.
\end{description}


\subsection{Figura}

A figura \ref{fig:graph} mostra uma figura. Quidquid latine dictum sit altum
viditur. The quick brown fox jumps over the lazy dog. Quidquid latine dictum
sit altum viditur. The quick brown fox jumps over the lazy dog.

\fig[.40]{graph}{Um grafo direcional}


\subsection{Tabela}

A tabela \ref{tab:tabela} mostra uma tabela. Quidquid latine dictum sit altum
viditur. The quick brown fox jumps over the lazy dog. Quidquid latine dictum
sit altum viditur. The quick brown fox jumps over the lazy dog.

\begin{table}[htbp]
	\caption{Tabela}
	\label{tab:tabela}
	\centering
	\begin{tabular}{|c|l|r|}
		\hline
		The 	&	Quick 	&	Brown	\\
		\hline
		Fox	&	Jumps	&	Over	\\
		The	&	Lazy	&	Dog	\\
		\hline 
	\end{tabular}
\end{table} 


\subsection{Citações (Referências)}

De acordo com \cite{DEAD:1666,BEEF:1234} este paragrafo exemplifica referências
(citações). Lorem ipsum dolor sit amet, consectetur adipisicing elit, sed do
eiusmod tempor incididunt ut labore et dolore magna aliqua. Ut enim ad minim
veniam, quis nostrud exercitation ullamco laboris nisi ut aliquip ex ea commodo
consequat. Duis aute irure dolor in reprehenderit in voluptate velit esse
cillum dolore eu fugiat nulla pariatur. Excepteur sint occaecat cupidatat non
proident, sunt in culpa qui officia deserunt mollit anim id est laborum.
Quidquid latine dictum sit altum viditur.


\section{Lorem Ipsum}

Lorem ipsum dolor sit amet, consectetur adipisicing elit, sed do eiusmod tempor
incididunt ut labore et dolore magna aliqua. Ut enim ad minim veniam, quis
nostrud exercitation ullamco laboris nisi ut aliquip ex ea commodo consequat.
Duis aute irure dolor in reprehenderit in voluptate velit esse cillum dolore eu
fugiat nulla pariatur. Excepteur sint occaecat cupidatat non proident, sunt in
culpa qui officia deserunt mollit anim id est laborum.


%\bibliographystyle{plain}
%\bibliographystyle{cell}
\bibliographystyle{alpha}
\bibliography{bib}


\end{document}


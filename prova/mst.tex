% LaTeX document template.
% Author: Rudy Matela
% e-mail: rudy.matela@gmail.com
% date: 2008-11-08 23:20


\documentclass[12pt,a4paper]{article}
\usepackage[utf8]{inputenc}
\usepackage[brazil]{babel}
\usepackage{graphicx}
\usepackage{indentfirst}
\usepackage{verbatim}
\usepackage{a4wide}
\usepackage{sty/macros}

\title{II NPC de PAA}
\author{ Renato Aguiar \and Tiago Justino }
\date{
    Prof. Marcos Negreiros \\
	Universidade Estadual do Ceará \\
	$\cdots$ \\
	\today
}


\begin{document}

\maketitle

\section{Projeto de Ambiente de Avaliação de Algoritmos de AGM}
\subsection{Descreva o seu ambiente de vizualização e cálculo de árvores e como
usá-lo.}

A programa possui uma interface de linha de comando (CLI) onde deve ser passado
como único parâmetro o nome do arquivo de entrada. Feito isso, todos os
algoritmos serão executados sobre o grafo contido no arquivo de entrada. Ao
final, a aplicação imprime na tela os resultados e gera arquivos de imagem para
o grafo original e para as árvores resultantes dos algoritmos.

\subsection{Mostre as diferenças entre os cálculos da AGM para os métodos de
PRIM, KRUSKAL e BORUVKA}

\begin{description}
\item[Prim:] É um algoritmo guloso que parte de um nó qualquer do grafo e vai
inserindo o nó mais próximo da árvore geradora parcial e que não forma ciclo.
\item[Kruskal:] É um algoritmo guloso que seleciona as arestas em ordem crescente,
eliminando aquelas que formam ciclos, até que todos os vértices tenham sido
alcançados.
\item[Boruvka:] O algoritmo de Boruvka começa com uma componente conexa para
cada nó e vai ligando as componentes através da aresta de menor custo entre
elas. Um característica interessante desse algoritmo é que o processamento de
cada componente pode ocorrer em paralelo.
\end{description}


\end{document}


% LaTeX document template.
% Author: Rudy Matela
% e-mail: rudy.matela@gmail.com
% date: 2008-11-08 23:20


\documentclass[12pt,a4paper]{article}
\usepackage[utf8]{inputenc}
\usepackage[brazil]{babel}
\usepackage{graphicx}
\usepackage{indentfirst}
\usepackage{verbatim}
\usepackage{a4wide}
\usepackage{sty/macros}

\title{II NPC de PAA}
\author{ Renato Aguiar \and Tiago Justino }
\date{
    Prof. Marcos Negreiros \\
	Universidade Estadual do Ceará \\
	$\cdots$ \\
	\today
}


\begin{document}

\maketitle

\section{Projeto de Ambiente de Avaliação de Algoritmos de AGM}
\subsection{Descreva o seu ambiente de vizualização e cálculo de árvores e como
usá-lo.}

A programa possui uma interface de linha de comando (CLI) onde deve ser passado
como único parâmetro o nome do arquivo de entrada. Feito isso, todos os
algoritmos serão executados sobre o grafo contido no arquivo de entrada. Ao
final, a aplicação imprime na tela os resultados e gera arquivos de imagem para
o grafo original e para as árvores resultantes dos algoritmos.

\subsection{Mostre as diferenças entre os cálculos da AGM para os métodos de
PRIM, KRUSKAL e BORUVKA}

\begin{description}
\item[Prim:] É um algoritmo guloso que parte de um nó qualquer do grafo e vai
inserindo o nó mais próximo da árvore geradora parcial e que não forma ciclo.
\item[Kruskal:] É um algoritmo guloso que seleciona as arestas em ordem crescente,
eliminando aquelas que formam ciclos, até que todos os vértices tenham sido
alcançados.
\item[Boruvka:] O algoritmo de Boruvka começa com uma componente conexa para
cada nó e vai ligando as componentes através da aresta de menor custo entre
elas. Um característica interessante desse algoritmo é que o processamento de
cada componente pode ocorrer em paralelo.
\end{description}

\section{Projeto de Árvores Geradoras com Restrições}

\subsection{Para o conjunto de dados de sua prova, mostre as topologias finais e
os custos das AGM usando dos métdos distintos de construção de AGM com restrição
de grau ($\delta=2$). Comente as diferençás encontradas.}

Em nosso trabalho implementamos os as adaptações dos algoritmos de prim (d-prim)
e kruskal para restrição de grau. Executamos para todas as entradas fornecidas e
obtivemos e pudemos perceber que blah blah blah.

As imagens dos grafos são apresentados nas figuras de
\ref{fig:5diamonds_grf_dcmst_kruskal_2} a \ref{fig:PEARN01_grf_dcmst_prim_2}.

\begin{itemize}
\item kruskal $\delta_{max}=2$
\begin{itemize}
\item 5diamonds.grf
\begin{itemize}
\item Tempo gasto =0.010000
\item Número de árvores = 3
\item Peso total = 1609
\end{itemize}
\item Euclid301.GRF
\begin{itemize}
\item Tempo gasto =0.000000
\item Número de árvores = 2
\item Peso total = 1842.6171507873030565
\end{itemize}
\item Euclid303.txt
\begin{itemize}
\item Tempo gasto =0.010000
\item Número de árvores = 2
\item Peso total = 3046.36832590716080566
\end{itemize}
\item Euclid306.txt
\begin{itemize}
\item Tempo gasto =0.000000
\item Número de árvores = 4
\item Peso total = 5308.06932867790105665
\end{itemize}
\item P20.GRF
\begin{itemize}
\item Tempo gasto =0.000000
\item Número de árvores = 23
\item Peso total = 326
\end{itemize}
\item P22.GRF
\begin{itemize}
\item Tempo gasto =0.000000
\item Número de árvores = 3
\item Peso total = 288
\end{itemize}
\item PEARN01.grf
\begin{itemize}
\item Tempo gasto =0.000000
\item Número de árvores = 1
\item Peso total = 11
\end{itemize}
\end{itemize}
\item Prim $\delta_{max}=2$
\begin{itemize}
\item 5diamonds.grf
\begin{itemize}
\item Tempo gasto =0.000000
\item Número de árvores = 7
\item Peso total = 1346
\end{itemize}
\item Euclid301.GRF
\begin{itemize}
\item Tempo gasto =0.000000
\item Número de árvores = 5
\item Peso total = 1670.07244168098226786
\end{itemize}
\item Euclid303.txt
\begin{itemize}
\item Tempo gasto =0.010000
\item Número de árvores = 2
\item Peso total = 3144.07714987317194171
\end{itemize}
\item Euclid306.txt
\begin{itemize}
\item Tempo gasto =0.000000
\item Número de árvores = 12
\item Peso total = 3756.46845685923577821
\end{itemize}
\item P20.GRF
\begin{itemize}
\item Tempo gasto =0.000000
\item Número de árvores = 41
\item Peso total = 72
\end{itemize}
\item P22.GRF
\begin{itemize}
\item Tempo gasto =0.000000
\item Número de árvores = 32
\item Peso total = 99
\end{itemize}
\item PEARN01.grf
\begin{itemize}
\item Tempo gasto =0.000000
\item Número de árvores = 1
\item Peso total = 10
\end{itemize}
\end{itemize}
\end{itemize}


\section{Anexo: Figuras}

\fig[.6]{5diamonds_grf_dcmst_kruskal_2}{5diamonds\_grf\_dcmst\_kruskal\_2}
\fig[.6]{5diamonds_grf_dcmst_prim_2}{5diamonds\_grf\_dcmst\_prim\_2}
\fig[.6]{Euclid301_GRF_dcmst_kruskal_2}{Euclid301\_GRF\_dcmst\_kruskal\_2}
\fig[.6]{Euclid301_GRF_dcmst_prim_2}{Euclid301\_GRF\_dcmst\_prim\_2}
\fig[.6]{Euclid303_txt_dcmst_kruskal_2}{Euclid303\_txt\_dcmst\_kruskal\_2}
\fig[.6]{Euclid303_txt_dcmst_prim_2}{Euclid303\_txt\_dcmst\_prim\_2}
\fig[.6]{Euclid306_txt_dcmst_kruskal_2}{Euclid306\_txt\_dcmst\_kruskal\_2}
\fig[.6]{Euclid306_txt_dcmst_prim_2}{Euclid306\_txt\_dcmst\_prim\_2}
\fig[.6]{P20_GRF_dcmst_kruskal_2}{P20\_GRF\_dcmst\_kruskal\_2}
\fig[.6]{P20_GRF_dcmst_prim_2}{P20\_GRF\_dcmst\_prim\_2}
\fig[.6]{P22_GRF_dcmst_kruskal_2}{P22\_GRF\_dcmst\_kruskal\_2}
\fig[.6]{P22_GRF_dcmst_prim_2}{P22\_GRF\_dcmst\_prim\_2}
\fig[.6]{PEARN01_grf_dcmst_kruskal_2}{PEARN01\_grf\_dcmst\_kruskal\_2}
\fig[.6]{PEARN01_grf_dcmst_prim_2}{PEARN01\_grf\_dcmst\_prim\_2}

\end{document}


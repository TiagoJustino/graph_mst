% LaTeX document template.
% Author: Rudy Matela
% e-mail: rudy.matela@gmail.com
% date: 2008-11-08 23:20


\documentclass[12pt,a4paper]{article}
\usepackage[utf8]{inputenc}
\usepackage[brazil]{babel}
\usepackage{graphicx}
\usepackage{indentfirst}
\usepackage{verbatim}
\usepackage{a4wide}
\usepackage{sty/macros}

\title{II NPC de PAA}
\author{ Renato Aguiar \and Tiago Justino }
\date{
    Prof. Marcos Negreiros \\
	Universidade Estadual do Ceará \\
	$\cdots$ \\
	\today
}


\begin{document}

\maketitle

\section{Projeto de Ambiente de Avaliação de Algoritmos de AGM}
\subsection{Descreva o seu ambiente de vizualização e cálculo de árvores e como
usá-lo.}

A aplicação possui uma interface web. Na página principal o usuário pode fazer
o upload do arquivo txt contendo o grafo. Na próxima tela serão mostrados os
resultados de todos os algoritmos disponíveis.

Para utilizar a aplicação é necessário rodar o servidor web através do script
rodar\_servidor.bat, que se encontra na pasta principal da aplicação.

Depois acessar a url http://localhost:8000/ e utilizar o sistema como descrito
anteriormente.

\subsection{Mostre as diferenças entre os cálculos da AGM para os métodos de
PRIM, KRUSKAL e BORUVKA}

\begin{description}
\item[Prim:] É um algoritmo guloso que parte de um nó qualquer do grafo e vai
inserindo o nó mais próximo da árvore geradora parcial e que não forma ciclo.
\item[Kruskal:] É um algoritmo guloso que seleciona as arestas em ordem crescente,
eliminando aquelas que formam ciclos, até que todos os vértices tenham sido
alcançados.
\item[Boruvka:] O algoritmo de Boruvka começa com uma componente conexa para
cada nó e vai ligando as componentes através da aresta de menor custo entre
elas. Um característica interessante desse algoritmo é que o processamento de
cada componente pode ocorrer em paralelo.
\end{description}

\section{Projeto de Árvores Geradoras com Restrições}

\subsection{Para o conjunto de dados de sua prova, mostre as topologias finais e
os custos das AGM usando dos métdos distintos de construção de AGM com restrição
de grau ($\delta=2$). Comente as diferençás encontradas.}

Em nosso trabalho implementamos os as adaptações dos algoritmos de prim (d-prim)
e kruskal para restrição de grau. Executamos para todas as entradas fornecidas e
pudemos perceber que a adaptação do kruskal tem uma pequena vantagem em alguns
poucos casos em relação ao prim, gerando um grafo final com um número um pouco
menor de árvores. Em 2 dos 7 grafos testados a adaptação do kruskal apresentou
um grafo final com uma aresta a mais que a adaptação do prim. Em questão de
performance os dois foram igualmente rápidos, executando em menos de 0.01
segundos.

As imagens dos grafos são apresentados nas figuras de
\ref{fig:5diamonds_grf_dcmst_kruskal_2} a \ref{fig:PEARN01_grf_dcmst_prim_2}.

\begin{itemize}
\item kruskal $\delta_{max}=2$
\begin{itemize}
\item 5diamonds.grf
\begin{itemize}
\item Tempo gasto = 0.010000
\item Número de árvores = 3
\item Peso total = 1609
\end{itemize}
\item Euclid301.GRF
\begin{itemize}
\item Tempo gasto = 0.000000
\item Número de árvores = 2
\item Peso total = 1842.6171507873030565
\end{itemize}
\item Euclid303.txt
\begin{itemize}
\item Tempo gasto = 0.010000
\item Número de árvores = 2
\item Peso total = 3046.36832590716080566
\end{itemize}
\item Euclid306.txt
\begin{itemize}
\item Tempo gasto = 0.000000
\item Número de árvores = 4
\item Peso total = 5308.06932867790105665
\end{itemize}
\item P20.GRF
\begin{itemize}
\item Tempo gasto = 0.000000
\item Número de árvores = 23
\item Peso total = 326
\end{itemize}
\item P22.GRF
\begin{itemize}
\item Tempo gasto = 0.000000
\item Número de árvores = 3
\item Peso total = 288
\end{itemize}
\item PEARN01.grf
\begin{itemize}
\item Tempo gasto = 0.000000
\item Número de árvores = 1
\item Peso total = 11
\end{itemize}
\end{itemize}
\item Prim $\delta_{max}=2$
\begin{itemize}
\item 5diamonds.grf
\begin{itemize}
\item Tempo gasto = 0.000000
\item Número de árvores = 3
\item Peso total = 1728
\end{itemize}
\item Euclid301.GRF
\begin{itemize}
\item Tempo gasto = 0.000000
\item Número de árvores = 3
\item Peso total = 1878.03842038914968879
\end{itemize}
\item Euclid303.txt
\begin{itemize}
\item Tempo gasto = 0.010000
\item Número de árvores = 2
\item Peso total = 3144.07714987317194171
\end{itemize}
\item Euclid306.txt
\begin{itemize}
\item Tempo gasto = 0.000000
\item Número de árvores = 5
\item Peso total = 4477.76746876996466442
\end{itemize}
\item P20.GRF
\begin{itemize}
\item Tempo gasto = 0.000000
\item Número de árvores = 23
\item Peso total = 340
\end{itemize}
\item P22.GRF
\begin{itemize}
\item Tempo gasto = 0.000000
\item Número de árvores = 3
\item Peso total = 289
\end{itemize}
\item PEARN01.grf
\begin{itemize}
\item Tempo gasto = 0.000000
\item Número de árvores = 1
\item Peso total = 10
\end{itemize}
\end{itemize}
\end{itemize}


\section{Testes Restrições}
\subsection{Para o conjunto de dados de Grafos Euclidianos que você dispõe,
verifique entre os métodos implementados com restrição de grau ($\delta_{max}=2,
3, 4$), os tempos de processamento, custos atingidos e topologia, qual o de
melhor de melhor desempenho, custo e comente seus resultados.}

.\begin{itemize}
\item kruskal $\delta_{max}=2$
\begin{itemize}
\item Euclid301.GRF
\begin{itemize}
\item Tempo gasto = 0.010000
\item Número de árvores = 2
\item Peso total = 1842.6171507873030565
\end{itemize}
\item Euclid303.txt
\begin{itemize}
\item Tempo gasto = 0.000000
\item Número de árvores = 2
\item Peso total = 3046.36832590716080566
\end{itemize}
\item Euclid306.txt
\begin{itemize}
\item Tempo gasto = 0.010000
\item Número de árvores = 4
\item Peso total = 5308.06932867790105665
\end{itemize}
\end{itemize}
\item kruskal $\delta_{max}=3$
\begin{itemize}
\item Euclid301.GRF
\begin{itemize}
\item Tempo gasto = 0.010000
\item Número de árvores = 1
\item Peso total = 1818.06589288847809766
\end{itemize}
\item Euclid303.txt
\begin{itemize}
\item Tempo gasto = 0.000000
\item Número de árvores = 1
\item Peso total = 2657.74898478985425974
\end{itemize}
\item Euclid306.txt
\begin{itemize}
\item Tempo gasto = 0.010000
\item Número de árvores = 1
\item Peso total = 3661.4300534816105989
\end{itemize}
\end{itemize}
\item kruskal $\delta_{max}=4$
\begin{itemize}
\item Euclid301.GRF
\begin{itemize}
\item Tempo gasto = 0.000000
\item Número de árvores = 1
\item Peso total = 1818.06589288847809766
\end{itemize}
\item Euclid303.txt
\begin{itemize}
\item Tempo gasto = 0.000000
\item Número de árvores = 1
\item Peso total = 2655.19712785249825139
\end{itemize}
\item Euclid306.txt
\begin{itemize}
\item Tempo gasto = 0.000000
\item Número de árvores = 1
\item Peso total = 3661.4300534816105989
\end{itemize}
\end{itemize}
\item prim $\delta_{max}=2$
\begin{itemize}
\item Euclid301.GRF
\begin{itemize}
\item Tempo gasto = 0.010000
\item Número de árvores = 3
\item Peso total = 1878.03842038914968879
\end{itemize}
\item Euclid303.txt
\begin{itemize}
\item Tempo gasto = 0.000000
\item Número de árvores = 2
\item Peso total = 3144.07714987317194171
\end{itemize}
\item Euclid306.txt
\begin{itemize}
\item Tempo gasto = 0.000000
\item Número de árvores = 5
\item Peso total = 4477.76746876996466442
\end{itemize}
\end{itemize}
\item prim $\delta_{max}=3$
\begin{itemize}
\item Euclid301.GRF
\begin{itemize}
\item Tempo gasto = 0.000000
\item Número de árvores = 1
\item Peso total = 1818.06589288847809766
\end{itemize}
\item Euclid303.txt
\begin{itemize}
\item Tempo gasto = 0.000000
\item Número de árvores = 1
\item Peso total = 2664.31141866200660931
\end{itemize}
\item Euclid306.txt
\begin{itemize}
\item Tempo gasto = 0.010000
\item Número de árvores = 1
\item Peso total = 3661.4300534816105989
\end{itemize}
\end{itemize}
\item prim $\delta_{max}=4$
\begin{itemize}
\item Euclid301.GRF
\begin{itemize}
\item Tempo gasto = 0.000000
\item Número de árvores = 1
\item Peso total = 1818.06589288847809766
\end{itemize}
\item Euclid303.txt
\begin{itemize}
\item Tempo gasto = 0.000000
\item Número de árvores = 1
\item Peso total = 2655.19712785249825139
\end{itemize}
\item Euclid306.txt
\begin{itemize}
\item Tempo gasto = 0.000000
\item Número de árvores = 1
\item Peso total = 3661.4300534816105989
\end{itemize}
\end{itemize}
\end{itemize}


\section{Anexo: Figuras}

\fig[.6]{5diamonds_grf_dcmst_kruskal_2}{5diamonds\_grf\_dcmst\_kruskal\_2}
\fig[.6]{5diamonds_grf_dcmst_prim_2}{5diamonds\_grf\_dcmst\_prim\_2}
\fig[.6]{Euclid301_GRF_dcmst_kruskal_2}{Euclid301\_GRF\_dcmst\_kruskal\_2}
\fig[.6]{Euclid301_GRF_dcmst_prim_2}{Euclid301\_GRF\_dcmst\_prim\_2}
\fig[.6]{Euclid303_txt_dcmst_kruskal_2}{Euclid303\_txt\_dcmst\_kruskal\_2}
\fig[.6]{Euclid303_txt_dcmst_prim_2}{Euclid303\_txt\_dcmst\_prim\_2}
\fig[.6]{Euclid306_txt_dcmst_kruskal_2}{Euclid306\_txt\_dcmst\_kruskal\_2}
\fig[.6]{Euclid306_txt_dcmst_prim_2}{Euclid306\_txt\_dcmst\_prim\_2}
\fig[.6]{P20_GRF_dcmst_kruskal_2}{P20\_GRF\_dcmst\_kruskal\_2}
\fig[.6]{P20_GRF_dcmst_prim_2}{P20\_GRF\_dcmst\_prim\_2}
\fig[.6]{P22_GRF_dcmst_kruskal_2}{P22\_GRF\_dcmst\_kruskal\_2}
\fig[.6]{P22_GRF_dcmst_prim_2}{P22\_GRF\_dcmst\_prim\_2}
\fig[.6]{PEARN01_grf_dcmst_kruskal_2}{PEARN01\_grf\_dcmst\_kruskal\_2}
\fig[.6]{PEARN01_grf_dcmst_prim_2}{PEARN01\_grf\_dcmst\_prim\_2}

\end{document}

